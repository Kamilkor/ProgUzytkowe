\documentclass[a4paper,12pt]{article}
\usepackage[MeX]{polski}
\usepackage[utf8]{inputenc}

%opening
%opening
\title{Wydzial Matematyki i Informatyki Uniwersytetu Warminsko-Mazurskiego}
\author{Kamil Kordaczuk}

\begin{document}

\maketitle

\begin{abstract}
\begin{center}
Wydzial Matematyki i Informatyki UWM
\end{center}
\end{abstract}

\section{Misja}
Misja wydzialu jest :
\begin{itemize}
 \item Ksztalcenie matematykow zdolnych do udzialu w rozwijaniu matematyki i jej stosowania w innych
dzialach wiedzy i w praktyce
 \item Ksztalcenie nauczycieli matematyki, nauczycieli matematyki z fizyka a takze nauczycieli informatyki
 \item Ksztalcenie profesjonalnych informatykow dla potrzeb gospodarki, administracji, szkolnictwa oraz zycia
spolecznego
 \item Nauczanie matematyki i jej dzialow specjalnych jak statystyka matematyczna, ekonometria,
biomatematyka, ekologia matematyczna, metody numeryczne; fizyki a w razie potrzeby i podstaw
informatyki na wszystkich wydzialach UWM.

 \end{itemize}
\section{Opis Kierunkow}

Na kierunku Informatyka prowadzone s� studia stacjonarne i niestacjonarne:

\begin{itemize}
 \item studia pierwszego stopnia � in�ynierskie (7 sem.), sp. in�ynieria system�w informatycznych, informatyka
og�lna
 \item studia drugiego stopnia � magisterskie (4 sem.), sp. techniki multimedialne, projektowanie system�w
informatycznych i sieci komputerowych
\end{itemize}

Na kierunku Matematyka prowadzone s� studia stacjonarne:

\begin{itemize}
\item studia pierwszego stopnia � licencjackie (6 sem.), sp. nauczanie matematyki, matematyka stosowana
\item studia drugiego stopnia � magisterskie (4 sem.), sp. nauczanie matematyki, matematyka stosowana
\end{itemize}

oraz studia niestacjonarne:

\begin{itemize}
\item studia drugiego stopnia � magisterskie (4 sem.), sp. nauczanie matematyki
\end{itemize}

Pa�stwowa Komisja Akredytacyjna w dniu 19 marca 2009r. oceni�a pozytywnie jako�� kszta�cenia na kierunku
Matematyka, natomiast w dniu 12 marca 2015r. oceni�a pozytywnie jako�� kszta�cenia na kierunku
Informatyka.


\end{document}